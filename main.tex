\documentclass{llncs}




\begin{document}




\title{Questionnaire generation for better polling}
\author{submitted for blind review}
\maketitle




\begin{abstract}




\end{abstract}




\section{Introduction}


%polling is important
Opinion poll is not uncommon in modern society. The first known example of an opinion poll was conducted nearly 200 years ago\footnote{}. It successfully predicted the outcome of United States presidency in 1824. Since then, it has been a convention to organize an opinion poll to measure public opinions and document the experiences of the public on a range of subjects, from marketing your favorite video games to important health and political issues. Polling results provide information for academics, researchers and government officials and help policymakers and other decision makers.


%conventional polling procedure
In an opinion poll, a series of questions are handed out to sample voters from all population or a particular group, depending on the target of opinion poll. The goal of an opinion poll is to extrapolate generalities based on the answers to the questions by the sample voters. Naturally, the accuracy of an opinion poll is highly affected by the designe of questionaire. As a result, opinion pollings are conventionally conducted by professionals with expertise. There are polling organizations in virtually every country with elections. Furthermore, all the major television networks, independently or working in conjunction with newspapers or magazines, operate opinion polling.


%motivation
Recently, online social media has attracted researchers to study public opinions~\cite{}. Many social media sites encourage the creation and exchange of ideas and opinions. In particular, debate forums are valuable platforms to quantify public opinions. As shown in Fig.~\ref{fig:ilustration}, a debate forum can be considered as a question bank, where on each question users are invited to vote positive and negative and comment when neccessary.


\begin{figure}[!ht]
\label{fig:illustration}
\centering
%\includegraphics[width=\textwidth]{.eps}
\caption{Illustrative example of questionire generation}
\end{figure}


%internet polling, problem definition(not formal)

In this paper, we explore the possibility of generating questionaire automatically from a question bank for any given opinion poll to a demographic population. The aim is to find a predefined number of yes/no questions that maximize the predictability on a given topic to a group of users with certain attributes. To the best of our knowledge, this is a novel task. However, estamating predictability of a set of questions is a very challanging problem. Even opinion polls generated by experts are not always reliable in predicting public opinions on an unseen topic. For instance, Donald Trump's election in 2016 signals the biggest and complete poll failure ever. Many factors can lead to erroneous polling results,  

%empirical study
Therefore our first contribution is to empirically study the factors that will influence predictability of a questionaire. The predictability of polling is strongly affected not only by the relevance of questionaire, but also by response bias, non response bias and coverage bias of the questionaire to the target population~\cite{}. 


%missing data


















%challenge










%contribution









%




\section{Related Work}
%Linking online media to polls
http://www.aaai.org/ocs/index.php/ICWSM/ICWSM10/paper/viewFile/1536/1842
A user-centric model of voting intention from Social Media




\section{Model}
 It is well regarded that the selection, ordering and wording of questions are critical~\cite{}. For simplicity, we don't consider the 
It is well established that asking enough questions to allow all aspects of an issue to be covered. 


\section{Missing Data Imputation}








\section{Experimental Analysis}




\section{Conclusion}
\end{document}
